\documentclass[brudnopis]{xmgr}
%\documentclass[xodstep]{xmgr}

%\defaultfontfeatures{Scale=MatchLowercase}
%\setmainfont[Numbers=OldStyle,Ligatures=TeX]{Minion Pro}
%\setsansfont[Numbers=OldStyle,Ligatures=TeX]{Myriad Pro}
% for fontspec version < 2.0
\setmainfont[Numbers=OldStyle,Mapping=tex-text]{Minion Pro}
\setsansfont[Numbers=OldStyle,Mapping=tex-text]{Myriad Pro}
%\setmonofont[Scale=0.75]{Monaco}

% Opcjonalnie identyfikator dokumentu 
% drukowany tylko z włączoną opcją 'brudnopis':
\wersja   {wersja wstępna [\ymdtoday]}

\author   {Dorian Sawa}
\nralbumu {186\,445}
\email    {dsawa@sigma.ug.edu.pl}

\title    {System weryfikacji jakości kodu w języku Scala}
\date     {2014}
\miejsce  {Gdańsk}

\opiekun  {dr W. Pawłowski}

% dodatkowe polecenia
%\renewcommand{\filename}[1]{\texttt{#1}}

\begin{document}

\begin{abstract}
 Scala to język programowania łączący dwie wydawałoby się różne techniki programowania - funkcyjnego i obiektowego. 
Rosnąca w ostatnim czasie jego popularność przyczyniła się do powstania wielu narzędzi pozwalających na testowanie 
programów napisanych w Scali. Niniejsza praca zgłębia temat wspomnianych narzędzi oraz ich wykorzystania jako podstawę 
systemu pozwalającego na automatyzację kontroli jakości kodu.
\end{abstract}
\keywords{Scala, 
 Play Framework, 
 Scalatest, 
 Scalacheck}

% tytuł i spis treści
\maketitle
%
% wstęp
\introduction Scala jest nowoczesnym językiem programowania działającym na wirtualnej maszynie Javy. Od chwili opublikowania jest sukcesywnie rozwijana, czego między innymi efektem jest fakt, że dzisiaj wiele firm korzysta z systemów napisanych w Scali. Systemy takie, napisane dla celów wewnętrznych, czy też komercyjnych spełniają swoje zadania z sukcesami. Przykładowymi firmami korzystającymi z technologii Scali mogą być znane marki takie jak Twitter, LinkedIn czy Intel. Chociaż Scala jest językiem całkowicie zorientowanym obiektowo, to posiada on także wszystkie elementy jakie można oczekiwać od języka funkcyjnego.

Przy wielu zaletach jakie ze sobą niesie możliwość programowania w Scali, udział procentowy tego języka na tle konkurencji wciąż jest stosukowo nieduży. Oczywiście proporcjonalnie do takiego rozkładu wygląda również sytuacja programistów Scali na rynku pracy. Na odwrócenie tych proporcji mogłaby mieć wpływ specjalna platforma pozwalająca przyspieszyć proces nauki wspomnianego wyżej języka. Nad tym zagadnieniem skupiać się będzie niniejsza praca.

Zaprojektowany i opisywany tu system powstał na bazie wielu nowoczesnych technologii oraz najważniejszych narzędzi do testów jednostkowych. Podstawą systemu jest Play Framework. Pozwala on na stosunkowo szybkie napisanie aplikacji internetowej zarówno w Scali jak i w Javie. Społeczność powstała wokół tych dwóch języków doprowadziła do powstania ogromnej liczby wszelakich bibliotek. Z powodzeniem mogą być one używane w aplikacji utworzonej przy użyciu tego narzędzia. 
Na szczególną uwagę zasługują Scalatest oraz Scalacheck. Pierwsza z nich to biblioteka do przeprowadzania testów jednostkowych. Umożliwia ona między innymi pisanie testów w wielu różnych stylach. Decyzja o tym jak ostatecznie wyglądać będzie test zależy od preferencji użytkownika. Scalacheck jest zaś instrumentem służącym do testowania funkcji, metod poprzez automatyczne wprowadzanie różnych argumentów. Tłem systemu jest nierelacyjna baza danych z wbudowanym mechanizmem do przechowywania plików - MongoDB. Interfejs użytkownika powstał przy pomocy AngularJS, który w przyjemny dla programisty sposób  potrafi dynamicznie zmieniać zawartość strony HTML warunkując ją od akcji użytkownika mających wpływ na stan modeli aplikacji.

\medskip Na początku swojej pracy postaram się szerzej przedstawić ideę będącą iskrę do stworzenia tego systemu. Pokażę, że dla innych języków programowania już funkcjonują pewne mechanizmy edukacyjne. Udowodnię ,że moja aplikacja jest jedyna w swoim rodzaju. Kolejnym krokiem będzie prezentacja języka Scala. Opowiem o mechanizmie aktorów, kombinacji języka funkcyjnego i obiektowego oraz innych nowych rzeczach jakie ze sobą niesie. Każde ze wspominanych wcześniej narzędzi zostanie również szerzej przedstawione. Każde z nich wywarło wpływ na ostateczny kształt opisywanej aplikacji. Ustosunkuję się do wyboru tych narzędzi spośród innych. Między innymi dlaczego zdecydowałem się na użycie nierelacyjnego MongoDB. I czemu uznałem, że standardowa funkcjonalność AngularJS to było dla mnie za mało. Ostatecznie przybliżę także sposób działania systemu oraz w jaki sposób użyte zostało jeszcze jedno narzędzie - SBT. Bezpośrednio związane z weryfikacją kodu uczestników systemu. 

\chapter{Idea systemu}

chapter chapterchapterchapterchapterchapter 

\section{Język Scala}

cytat ~\cite[s.~123]{Elmasri:2002:CMC}. 
Język ten stworzony został przez Martina Odersky'ego, a upubliczniony w 2004 roku.

    
\section{Środki do osiągnięcia celu}

aaaaaaaaaaaaaaaaaaaa

\subsection{Scalacheck}

aaaaaaaaaaaaaaaaaaaa

\subsection{Scalatest} 

aaaaaaaaaaaaaaaaaaaa

scalatest jest git
      
\chapter{Mechanizm testów jednostkowych}

test
    
\section{Testowanie w języku Scala}

scala scala test test 
      
\section{Narzędzie Scalatest w szczegółach}

scalatest testscala

\chapter{System weryfikacji jakości kodu Scala}

\section{Struktura ogólna}

\section{Scenariusze użycia}

\section{Użyte technologie}

\subsection{SBT - Scala Build Tool}

\subsection{Play Framework}

\subsection{Scalatest}

\subsection{AngularJS}

czemu angular i testy w angularze tez

\subsubsection{Testy w AngularJS}

\subsection{MongoDB}

% zakończenie 
\summary
Możliwości, jakie stoją przed archiwum prac magisterskich opartych na
XML-u, są ograniczone jedynie czasem, jaki należy poświęcić na pełną
implementację systemu. Nie ma przeszkód technologicznych do stworzenia
co najmniej równie doskonałego repozytorium, jak ma to miejsce w
przypadku ETD. Jeżeli chcemy w pełni uczestniczyć w rozwoju nowej ery
informacji, musimy szczególną uwagę przykładać do odpowiedniej
klasyfikacji i archiwizacji danych. Sądzę, że język XML znacznie to
upraszcza.

% załączniki (opcjonalnie):
\appendix
\chapter{Tytuł załącznika jeden}

Treść załącznika jeden.

\chapter{Tytuł załącznika dwa}

Treść załącznika dwa.

% literatura (obowiązkowo):
\bibliographystyle{unsrt}
\bibliography{xml}

% spis tabel (jeżeli jest potrzebny):
\listoftables

% spis rysunków (jeżeli jest potrzebny):
\listoffigures

\oswiadczenie

\end{document}
