\documentclass[brudnopis]{xmgr}

%\defaultfontfeatures{Scale=MatchLowercase}
%\setmainfont[Numbers=OldStyle,Ligatures=TeX]{Minion Pro}
%\setsansfont[Numbers=OldStyle,Ligatures=TeX]{Myriad Pro}
% for fontspec version < 2.0
\setmainfont[Numbers=OldStyle,Mapping=tex-text]{Minion Pro}
\setsansfont[Numbers=OldStyle,Mapping=tex-text]{Myriad Pro}
%\setmonofont[Scale=0.75]{Monaco}

% Opcjonalnie identyfikator dokumentu 
% drukowany tylko z włączoną opcją 'brudnopis':
\wersja   {wersja wstępna [\ymdtoday]}

\author   {Dorian Sawa}
\nralbumu {186\,445}
\email    {dsawa@sigma.ug.edu.pl}

\title    {System weryfikacji jakości kodu w języku Scala}
\date     {2014}
\miejsce  {Gdańsk}

\opiekun  {dr W. Pawłowski}

% dodatkowe polecenia
%\renewcommand{\filename}[1]{\texttt{#1}}

\begin{document}

\begin{abstract}
 Scala to język programowania łączący dwie wydawałoby się różne techniki programowania - funkcyjnego i obiektowego. 
Rosnąca w ostatnim czasie jego popularność przyczyniła się do powstania wielu narzędzi pozwalających na testowanie 
programów napisanych w Scali. Niniejsza praca zgłębia temat wspomnianych narzędzi oraz ich wykorzystania jako podstawę 
systemu pozwalającego na automatyzację kontroli jakości kodu.
\end{abstract}
\keywords{Scala, 
 Play Framework, 
 Scalatest, 
 Scalacheck}

% tytuł i spis treści
\maketitle
%
% wstęp
\introduction

wstep wstep wstep

\chapter{Idea systemu}

chapter chapterchapterchapterchapterchapter 

\section{Język Scala}

cytat ~\cite[s.~123]{Elmasri:2002:CMC}. 
gites
    
\section{Środki do osiągnięcia celu}

aaaaaaaaaaaaaaaaaaaa

\subsection{Scalacheck}

aaaaaaaaaaaaaaaaaaaa

\subsection{Scalatest} 

aaaaaaaaaaaaaaaaaaaa

scalatest jest git
      
\chapter{Mechanizm testów jednostkowych}

test
    
\section{Testowanie w języku Scala}

scala scala test test 
      
\section{Narzędzie Scalatest w szczegółach}

scalatest testscala

\chapter{System weryfikacji jakości kodu Scala}

\section{Struktura ogólna}

\section{Scenariusze użycia}

\section{Użyte technologie}

\subsection{SBT - Scala Build Tool}

\subsection{Play Framework}

\subsection{Scalatest}

\subsection{AngularJS}

czemu angular i testy w angularze tez

\subsubsection{Testy w AngularJS}

\subsection{MongoDB}

% zakończenie 
\summary
Możliwości, jakie stoją przed archiwum prac magisterskich opartych na
XML-u, są ograniczone jedynie czasem, jaki należy poświęcić na pełną
implementację systemu. Nie ma przeszkód technologicznych do stworzenia
co najmniej równie doskonałego repozytorium, jak ma to miejsce w
przypadku ETD. Jeżeli chcemy w pełni uczestniczyć w rozwoju nowej ery
informacji, musimy szczególną uwagę przykładać do odpowiedniej
klasyfikacji i archiwizacji danych. Sądzę, że język XML znacznie to
upraszcza.

% załączniki (opcjonalnie):
\appendix
\chapter{Tytuł załącznika jeden}

Treść załącznika jeden.

\chapter{Tytuł załącznika dwa}

Treść załącznika dwa.

% literatura (obowiązkowo):
\bibliographystyle{unsrt}
\bibliography{xml}

% spis tabel (jeżeli jest potrzebny):
\listoftables

% spis rysunków (jeżeli jest potrzebny):
\listoffigures

\oswiadczenie

\end{document}
